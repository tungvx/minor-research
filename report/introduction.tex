\label{sec:introduction}

Semantic parsing is a very important process in Natural Language Processing. It converts a natural language sentence into a special meaning representation so that computer programs can read and process it. An effective way to do this task is using machine learning to build a model of relationship between natural language structure and formal representation structure. Unfortunately, most of the current approaches require very large amount of fully annotated data in order to obtain a good model. Annotated data means that for each input sentence of natural language, the corresponding representation is also provided. State-of-the-art methodologies often use statistical analysis to build the model and thus generally ill-perform with unseen data. Therefore, in order to get good results, they need to collect a large amount of annotated corpus. The annotation, however, is almost prepared manually and thus is difficult and time consuming. \cite{Zelle:1996:LPD:1864519.1864543}, \cite{Tang:2001:UMC:645328.650015}, \cite{Zettlemoyer05learningto}, \cite{Ge:2005:SSP:1706543.1706546}, \cite{Zettlemoyer07onlinelearning} and \cite{Wong07learningsynchronous} are examples of works in this category.

Recently, \citeauthor{Clarke:2010:DSP:1870568.1870571} in \cite{Clarke:2010:DSP:1870568.1870571} implemented a new learning paradigm aimed at alleviating the supervision burden. The algorithm is able to predict complex structures which only rely on a binary feedback. Borrowing the idea from these authors, we developed a question answering system for Vietnamese with suitable feature computations.

\subsection*{Related Work}
There has been not many works in Vietnamese semantic parsing. \citeauthor{Nguyen:2009:VQA:1681518.1683170} in \cite{Nguyen:2009:VQA:1681518.1683170} developed a question answering system for Vietnamese. The system enable users to query an ontological knowledge base using pattern matching. Although this kind of semantic parsing is simple and their testing data is small, the experiment shows promising results. 

The rest of this paper is organized as follow. Section \ref{sec:c-u} describes how natural language sentences are represented formally in our semantic parser. Section \ref{sec:model} illustrates the way we maintain the parsing model. We present our experiment result and discuss it in Section \ref{sec:experiment}. Finally, conclusion and future works are drawn in Section \ref{sec:conclusion}.