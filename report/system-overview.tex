\label{sec:system-overview}
The goal of the system is to answer the questions of users properly. The core of our system is semantic parsing. As presented, given an input sentence $x$, our semantic parser tries to make a mapping $y$ between tokens and functions. Then from the mapped functions, it will search for a composition $z$ of them for final logical form. The problem is that for each $x$, there may be some acceptable $y$; and for each $y$, there are some acceptable $z$. But semantic parser is required to choose only one pair of $y$ and $z$. 

Our system will consider all the possible cases and putting scores on relations between $x$ and $y$, $y$ and $z$. After that, it decides the best meaning representation in terms of the total scores of each pair of  mapping and composition. More precisely, the prediction function is as follow:
\begin{equation}
\label{prediction-1}
\hat{y}, \hat{z} = arg \ max_{y,z} (score(x, y) + score(y,z) )
\end{equation}
We represent the $score$s by features vectors and weight vector. Let $\theta_1(x, y)$ and $\theta_2(y,z)$ be the feature vectors that represent relations between $x$ and $y$, $y$ and $z$ respectively. $w$ is the feature vector. Equation \ref{prediction-1} now become:
\begin{equation}
\label{prediction-2}
\hat{y}, \hat{z} = arg \ max_{y,z} w^T(\theta_1(x, y) + \theta_2(y,z) )
\end{equation}
The feature vectors and weight vector have the same size. In our system, their size are $3$. Solving the prediction in equation \ref{prediction-2} is done by Linear Programming solver. 

%Weight vector is obtained by machine learning. We apply two learning methods mentioned \cite{Clarke:2010:DSP:1870568.1870571} which are "Direct approach" and "Aggressive Approach".
%
%\section{Direct Approach}
%The training data contains pairs of Vietnamese queries and the corresponding answer. After our semantic parser outputs the final result of function composition. This logical form is then translated into the Prolog query. The newly created query is executed with the Geoquery database. The obtained result is compared with the answer. Thus we can receive the feedback whether a result generated by our system is true or not. This scenario is quite practical; because when a user observe the response, he can judge the answer. 
%
%\section{Aggressive Approach}