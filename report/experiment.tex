\chapter{Experiment}
We use the Geoquery domain for evaluating our system. It consists of a database and Prolog query language for U.S. geographical facts. The corpus has 880 English tokenized queries; each query is paired with a Prolog query. We use Google Translate API to translate the English queries into Vietnamese ones. The translation contains many errors as the result of some word-word translations. We then manually correct each query to produce meaningful Vietnamese queries. 

Following the experiment in \cite{Clarke:2010:DSP:1870568.1870571}, we randomly select 250 queries for training, 250 other queries for testing from the above 880 queries. Our experiment is used to answer the following doubt:

\begin{enumerate}
  \item What are the effects of dynamic compositional preferences learning?
  \item How important are two learning approaches?
\end{enumerate}

\section{What are the effects of dynamic compositional preferences learning?}
We set up the experiment that when calculating the score for functions composition, the second feature is the same as the first one. That means they are both computed based on the distance of two tokens. The result is described in table \ref{d-c-p-eff}. We can see that applying this strategy improves the accuracy of the system by $4.4\%$ and $3.2\%$ for training data and testing data respectively. We also may wonder that if some set of questions is asked many times by users, how does this strategy perform? The last column of table \ref{d-c-p-eff} shows that if we repeatedly run the training data 10 times, then the final result on the training data is very surprising, $62\%$ is reported. In addition, if we run on testing data after running 10 times on training data, the accuracy for testing data is not significantly affected. It decreases by 1.6\%. 
\begin{table}[h] 
	\begin{center}
	    \begin{tabular}{| p{5cm} | c | c |}
	    \hline
	    Algorithm & Training set & Testing set \\ \hline
		No dynamic learning & 53.6\% & 51.6\%  \\ \hline
	    Dynamic learning & 58\% & 54.8\%  \\ \hline
	    Dynamic learning with repetition on training data & 62\% & 53.2\%  \\
	    \hline
	    \end{tabular}        
	\end{center}
	\scriptsize
	\caption{Accuracy of model on traing data and testing data. "No dynamic learning" means that we do not learn compositional preferences during running time of system. "Dynamic learning" means that we apply this kind of learning. in "Dynamic learning with repetition on training data", we run the system 10 times on training data; then testing data is also experimented}
    \label{d-c-p-eff}
\end{table}